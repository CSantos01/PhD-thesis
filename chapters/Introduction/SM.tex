\section{The Standard Model of particle physics}\label[section]{sec:SM}

The \gls{SM} provides mathematical tools to describe the interactions between elementary particles, which can be \textit{fermions}, with half-odd spin, or \textit{bosons}, with integer spin.
The elementary bosons, or \textit{gauge bosons}, act as mediators of the fundamental interactions: \textit{electromagnetic}, \textit{weak} and \textit{strong} interactions.
The elementary fermions are, in this theory, 12 particles (and 12 anti-particles) forming multiplets of the groups SU(3)$_C$, SU(2)$_L$ and U(1)$_Y$.

The $SU(2)_L \otimes U(1)_Y$ gauge group describes the electroweak interaction, mediated respectively by the $\left(W^i\right)_{i\in\llbracket1,3\rrbracket}$ and $B$ bosons, acting on the \textit{weak isospin} $T$ and the \textit{weak hypercharge} $Y$.
At lower energy, this symmetry is spontaneously broken by the Higgs mechanism, leading to the appearance of the $W^{\pm}$ and $Z$ bosons, and the photon $\gamma$, which fields are defined as such:

\begin{equation*}
    W^{\pm} = \frac{1}{\sqrt{2}}\left(W^1 \mp iW^2\right),
    \quad
    Z = W^3\cos\theta_W - B\sin\theta_W,
    \quad
    A = W^3\sin\theta_W + B\cos\theta_W
\end{equation*}

where $\theta_W$ is the weak, or Weinberg, angle.
One also need to redefine the generator of $U(1)_Y$ as $Q = T_3 + \frac{Y}{2}$, which is the electric charge operator.
Additionnaly, this symmetry breaking mechanism leads to the appearance the Higgs boson $H$, which is the only scalar elementary particle in the \gls{SM}.

The SU(3)$_C$ gauge group describes the strong interaction, and act only on particles with a \textit{colour charge} $C\in\{R,G,B,\Bar{R},\Bar{G},\Bar{B}\}$. This group being of dimension 8, it has 8 gauge bosons, called \textit{gluons} $\left(g_i\right)_{i\in\llbracket0,8\rrbracket}$.
Elementary fermions can then be separated in two categories:

    \begin{itemize}
        \item \textit{Quarks}, with a colour charge, which are grouped in three generations, each containing two quarks with electric charge $Q=\frac{2}{3}$ and one with $Q=-\frac{1}{3}$, as follows:
        \begin{equation*}
            \begin{pmatrix}
                u \\
                d
            \end{pmatrix}
            \quad
            \begin{pmatrix}
                c \\
                s
            \end{pmatrix}
            \quad
            \begin{pmatrix}
                t \\
                b
            \end{pmatrix}
        \end{equation*}
        and their anti-particle equivalents. 
        As they interact with the strong nuclear force, quarks are never observed in isolation, but always in bound states called hadrons (except for the top quark because of its mass), since free particles must always have a "null" colour charge.
        Particles composed of two quarks are called mesons, and those composed of three quarks are called baryons.
        
        \item \textit{Leptons}, without a colour charge, which are also grouped in three generations, each containing a charged lepton and a neutrino, as follows:
        \begin{equation*}
            \begin{pmatrix}
                \nu_e \\
                e^-
            \end{pmatrix}
            \quad
            \begin{pmatrix}
                \nu_\mu \\
                \mu^-
            \end{pmatrix}
            \quad
            \begin{pmatrix}
                \nu_\tau \\
                \tau^-
            \end{pmatrix}
        \end{equation*}
    \end{itemize}
    and their anti-particle equivalents.

    All of the fermions interact weakly, and charged fermions also interact electromagnetically.
    From the way they act under $SU(2)_L$, one can write the fermions as \textit{weak-isospin doublets}
    \begin{equation*}
        \begin{pmatrix}
            \nu_e \\
            e^-
        \end{pmatrix}_L
        \quad
        \begin{pmatrix}
            \nu_\mu \\
            \mu^-
        \end{pmatrix}_L
        \quad
        \begin{pmatrix}
            \nu_\tau \\
            \tau^-
        \end{pmatrix}_L
        \quad
        \begin{pmatrix}
            u' \\
            d'
        \end{pmatrix}_L
        \quad
        \begin{pmatrix}
            c' \\
            s'
        \end{pmatrix}_L
        \quad
        \begin{pmatrix}
            t' \\
            b'
        \end{pmatrix}_L
    \end{equation*}

    and \textit{weak-isospin singlets}
    \begin{equation*}
        e^-_R
        \quad
        \mu^-_R
        \quad
        \tau^-_R
        \quad
        u'_R
        \quad
        c'_R
        \quad
        t'_R
        \quad
        d'_R
        \quad
        s'_R
        \quad
        b'_R
    \end{equation*}

with the $L$ and $R$ subscripts indicating the chirality of the particles, left-handed or right-handed, respectively.
The right-handed neutrinos are not included in the \gls{SM}.

The primes in the expressions above indicate that the quarks' "\textit{weak}" eigenstates are not the same as their "\textit{mass}" eigenstates. 
The two bases are related by the unitary \gls{CKM} matrix, $V_{\text{CKM}} \in M_3\left(\mathbb{C}\right)$:

\begin{equation}
    \begin{pmatrix}
        d' \\
        s' \\
        b'
    \end{pmatrix}
    =
    \begin{pmatrix}
        V_{ud} & V_{us} & V_{ub} \\
        V_{cd} & V_{cs} & V_{cb} \\
        V_{td} & V_{ts} & V_{tb}
    \end{pmatrix}
    \begin{pmatrix}
        d \\
        s \\
        b
    \end{pmatrix}
\end{equation}

Due to physical reasons, we need only three mixing angles $\left(\theta_{12}, \theta_{13}, \theta_{23}\right) \in \mathbb{R}^3$ and one phase $\delta_{13} \in \mathbb{R}$ in order to fully parametrize this matrix.
The phase is interpreted as the \textit{CP-violating} phase and is an important parameter of the \gls{SM} to measure because of its implication in phenomena such as the \textit{matter-antimatter asymmetry}.

The unitary of the \gls{CKM} matrix implies that:

\begin{align}
    \sum_{i \in \{u, c, t\}} V_{ij}V_{ik}^* =  \delta_{jk}, \quad \left(j, k\right) \in \{d, s, b\}^2 \label[equation]{eq:CKM_unitarity} \\ 
    \sum_{i \in \{d, s, b\}} V_{ji}V_{ki}^* = \delta_{jk}, \quad \left(j, k\right) \in \{u, c, t\}^2
\end{align}

where $\delta$ is the Kronecker symbol.

Developping the sum of \cref{eq:CKM_unitarity} for $j = d$ and $k = b$ gives:

\begin{align}
    V_{ud}V_{ub}^* + V_{cd}V_{cb}^* + V_{td}V_{tb}^* = 0 \\
    \Longleftrightarrow 1 + \frac{V_{cd}V_{cb}^*}{V_{ud}V_{ub}^*} + \frac{V_{td}V_{tb}^*}{V_{ud}V_{ub}^*} = 0
\end{align}

supposing that $V_{ud}V_{ub}^* \neq 0$ (which is true for physical reasons).
The last relation describe the \textit{unitarity triangle} in the complex plane, with two of its vertices at $C = (0, 0)$ and $B = (1, 0)$ respectively.
The last vertex is defined to be at $A = (\Bar{\rho}, \Bar{\eta}) \in \mathbb{R}^2$
Consequently, the lenghts of the triangle sides are:

\begin{align}
    \Bar{AB} &= \abs{\frac{V_{td}V_{tb}^*}{V_{cd}V_{cb}^*}} \\
    \Bar{AC} &= \abs{\frac{V_{ud}V_{ub}^*}{V_{cd}V_{cb}^*}} \\
    \Bar{BC} &= 1
\end{align}

and the angles, also parametrized by the \gls{CKM} matrix elements, are:

\begin{align}
    \alpha = \arg\left(\frac{V_{cd}V_{cb}^*}{V_{td}V_{tb}^*}\right) \\
    \beta =  \arg\left(\frac{V_{td}V_{tb}^*}{V_{cd}V_{cb}^*}\right) \\
    \gamma = \arg\left(\frac{V_{ud}V_{ub}^*}{V_{cd}V_{cb}^*}\right)
\end{align}

Hence, the measurement of the sides and angles of the unitarity triangle allows to test the \gls{SM} and search for \gls{NP}.
